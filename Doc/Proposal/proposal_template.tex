\documentclass[a4paper, 12pt]{article}
%\usepackage{enumerate}
\usepackage{graphicx}
 \usepackage{url} % allow url in bibtex

\title{CS412 - Project Plan\\ Face expression recognition (1)}
\date{\today}

\begin{document}

\begin{center} 
\large VNUHCM - University of Science\\
Faculty of Information Technology\\
Advanced Program in Computer Science
\end{center}

\begingroup
\let\newpage\relax
\maketitle
\endgroup

\textbf{Group members:}
\begin{enumerate}
	\item 1351040 : Thai Thien
	\item 1351059 : Ho Ngoc Huynh Mai
\end{enumerate}

\section{Introduction}

We choose to do Face expression recognition. Our implement will based on paper Robust Facial Expression Classification Using Shape and Appearance Features of SL Happy and Aurobinda Routray \cite{7050661}. We adopt the preprocessing process, the feature extraction. But we do not intend to work on extract active patches. For the model, we plan to implement 2 model, a Support Vector Machine (as in \cite{7050661}) and a simple Neural network.      

\section{Project Details}
The purpose of this project is implement the system to recognize facial expression, including, but not limited to anger, disgust, fear, happiness, sadness and surprise. \\
Input: A face of someone. \\
Output: The facial expression. \\ 

We plan to do the project in 2 state. At first, we follow the tutorial, use pre-implement function (such as cv2.goodFeaturesToTrack()) and simple model to get the code run and solve the problem. Then, we implement LBP, PHOG, SVM as in \cite{7050661} as well as improve the neural network.
\subsection{Theory}
Face expression recognition is a machine learning classification problems.   
\subsection{Methodology}
\subsubsection{Preprocessing}
\begin{itemize}
	\item Noise filter using Gaussian kernel
	\item Face detection using Haar Cascades (have been implemented in Opencv2) \cite{Haar}
	\item Extract the face then resize to 96x96
\end{itemize}
\subsubsection{Feature selection}
	This step we select feature to represent image. 
	For basic level, we try to implement the basic feature selection base on Open CV tutorial, such as Harris Corner Detection, Shi-Tomasi Corner Detector, Good Features to Track. 
	Then, we try to implement Local binary patterns (LBP) and pyramid of histogram of gradients (PHOG). 
	
\subsubsection{The model}
	The model is a classifier model, which take feature vector as a input and output the class it belong to. There are 6 classes anger, disgust, fear, happiness, sadness and surprise.
	
	For basic level, we implement a neural net, multi-layer perceptrons as the tutorial \cite{mlp}.
	
	Then, we implement the One-Against-One Support Vector Machine. We need total 15 OAO SVM for 6 classes.
	

\subsection{Data, Scenarios, and Models}
Dataset: Cohn-Kanade (CK and CK+) database \cite{5543262}

%\subsubsection*{Policy applications: \ldots}
\subsection{Time plan for project}
week 1: Get used to OpenCV\\
week 2: Do the preprocessing \\
week 3: Feature selection (simple one)\\
week 4: Implement the very simple classifier. \\
Week 5: Try to put thing together and make it run. \\
Week 6: Improve feature selection (using idea in \cite{7050661} ) \\
Week 7: Implement SVM as in \cite{7050661}\\

\bibliographystyle{ieeetr}
\bibliography{Bibtex}

\end{document}

